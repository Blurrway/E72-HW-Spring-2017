\documentclass[12pt,letterpaper]{hmcpset}
\usepackage[margin=1in]{geometry}
\usepackage{graphicx}
\usepackage{amsmath}
\usepackage{afterpage}
% info for header block in upper right hand corner
\name{}
\class{Engineering 72}
\assignment{Homework 2}
\duedate{Due 3 February 2017}

\newcommand{\pn}[1]{\left( #1 \right)}
\newcommand{\abs}[1]{\left| #1 \right|}
\newcommand{\bk}[1]{\left[ #1 \right]}
\newcommand{\vc}[1]{\left\langle #1 \right\rangle}
\newcommand{\pder}[2]{\frac{\partial #1}{\partial #2}}

\newcommand{\norm}[1]{\lvert #1 \rvert}

% set numbering style for enumerated lists to be of form (a), (b), (c), etc.
\renewcommand{\labelenumi}{{(\alph{enumi})}}

% command to make 2in wide image centered on page
\newcommand{\diagram}[2]{\begin{center}\includegraphics[width=#2in, keepaspectratio]{#1}\end{center}}

\begin{document}
	
\problemlist{1, 2, 3, 4}

\textbf{Reminder}: Explain your solution process in complete sentences. You may lose points due to
poor presentation. Review policy on collaboration in the class syllabus. If you use any outside
resources be sure to cite your source.

\begin{problem}[1]
	 As shown in Figure.1, the signal $x(t)$ is approximated by the two building blocks, i.e. $x(t) \cong \widehat{x}(t) = a_1 \phi_1 (t) + a_2 \phi_2 (t)$.
	 \begin{enumerate}
	 	\item Determine the coefficients $a_1$ and $a_2$ such that integrated square error (ISE) is
	 	minimized. Sketch the approximated signal $\widehat{x}(t)$. Label the relevant amplitudes and
	 	times.
	 	
	 	\item Consider a circuit system that is linear and time-invariant. If the response to the input
	 	waveform $\phi_1 (t)$ is $y_1 (t)$, and the response to the input waveform $\phi_2 (t)$ is $y_2 (t)$, what
	 	is the response of the system to $\widehat{x}(t)$? Please explain your results.
	 	
	 	\item Sketch another non-trivial basis function $\phi_3 (t)$, defined in the region $0 \leq t \leq 1$, that is orthogonal to both $\phi_1 (t)$ and $\phi_2 (t)$. Label the relevant amplitudes and times. Let $x(t) \cong \widehat{x}(t) = a_1 \phi_1 (t) + a_2 \phi_2 (t) + a_3 \phi_3 (t)$, what are the new coefficients? Describe the advantage of using orthogonal basis functions.
	 \end{enumerate} 
	 \diagram{Figure_1}{4.75}
\end{problem}

\begin{solution}
\vfill
\end{solution}

\afterpage{\null\newpage}

\begin{problem}[2] 
	A Hadamard matrix consists of elements $\pm$1 and has the property that its rows are
	orthogonal to each other (recall that if two vectors are orthogonal their dot product is zero).
	Starting with a 2x2 matrix $[H_2]$, a matrix of $[H_{2n}]$ of size 2nx2n can be constructed
	recursively using the equation given below.
	$$ [H_2]=\begin{bmatrix} 1 & 1 \\ 1 & -1 \end{bmatrix}
	   \hspace{0.5in}
	   [H_{2n}]=\begin{bmatrix} [H_n] & [H_n] \\ [H_n] & -[H_n] \end{bmatrix} $$
	\begin{enumerate}
		\item Construct $[H_4]$ and $[H_8]$.
		
		\item If $c_{ik}$ denotes the elements in the i$^{th}$ row and k$^th$ column of $[H_4]$, sketch the following
		waveforms:
		$$ \phi_i (t) = \sum_{k=1}^4 c_{ik}p(t-(k-1)T),\hspace{2ex} i = 1,2,3,4$$
		where $p(t)$ is a pulse of unit amplitude on the interval $0 \leq t < T$, and $T$ is an arbitrary time
		interval. The waveforms $\phi_i (t)$ are known as \textbf{Walsh functions} and among other things are used to
		separate channels in Code Division Multiple Access (CDMA) cell phones. When the codes used
		in CDMA system are orthogonal to each other, interference from other users (codes) can be
		minimized at the detection system when the user channels are properly synchronized.
		
		\item Are the $\phi_i (t)$ of part (b) orthogonal to each other? Why or why not? Discuss the
		relationship of result to the orthogonal nature of the pair of the row vectors.
		
		\item Determine the coefficients $c_k$ that minimize the ISE in the series approximation of the
		signal shown in Figure 2, i.e.
		$$ x(t) \approx \widehat{x}(t) = \Sigma_{k=1}^4 c_k \phi_k (t) $$
	\end{enumerate}
	\diagram{Figure_2}{2.5}
\end{problem}

\begin{solution}
	\vfill
\end{solution}

\afterpage{\null\newpage}

\newpage

\begin{problem}[3]
	As discussed in class, Fourier Series can be used to represent a periodic signal because the
	complex exponential building blocks are also periodic with the same $T_0$. In order to apply
	the Fourier Series method to a non-periodic signal $x(t)$ over a given time interval, say $T$,
	we can instead construct the periodic extension of the signal as $x_p (t)$, by choosing a $T_0$
	that is something larger than $T$ (i.e. including all the interested time).
	
	Mathematically,
	$$ x_p (t) = \sum_{n=-\infty}^{\infty} x(t-nT_0) $$
	\begin{enumerate}
		\item For the signal $x(t)$ shown in Figure 3,  determine the expression for the Fourier
		coefficients of $x_p (t)$ if we choose $T_0 = T$. We assume the Fourier Series form is given
		as $x(t) \cong \widehat{x}(t) = \Sigma_{k=-K}^K c_k \phi_k (t)$, where $\phi_k (t) = e^{jk\omega_0 t}$ , and $\omega_0 = \frac{2\pi}{T_0}$.
		
		\item Create a Matlab function to generate the graphs (include a couple periods at least) of
		$\widehat{x}(t)$ for A=2, T=1, T$_0$=1 and K=10, 20 and 50. Estimate the ISE for each choice of
		K and display the result on the graph.
	\end{enumerate}
	\diagram{Figure_3}{2.5}
\end{problem}

\begin{solution}
	\vfill
\end{solution}

\afterpage{\null\newpage}
\newpage

\begin{problem}[4]
	Let the pulse train $x(t)$ used in the class example pass through a RC circuit system below.
	The voltage across the capacitor C be the output $y(t)$. Assume A=1 volt, $\tau$ =
	0.25 msec, $T_0$ = 1 msec. R=0.5 k$\Omega$, C=$\frac{1}{2\pi}$ $\mu$F. \textbf{From E79}, you have learned that each
	complex exponential component $c_k e^{jk\omega_0 t}$ of the pulse train after the LTI system is scaled
	by $H(jk\omega_0)$ or its frequency response function at the corresponding exciting frequency of
	$k\omega_0$.	
	\begin{enumerate}
		\item Use your Matlab code from the class to sketch the input $x(t)$ and output waveform
		$y(t)$ for comparison. You should decide on a proper number of terms $K$ to use. 
		
		\item Explain your results and what the RC circuit does to the input signal.
	\end{enumerate}
	\diagram{Circuits_4}{4}
\end{problem}

\begin{solution}
	\vfill
\end{solution}

\end{document}