\documentclass[12pt,letterpaper]{hmcpset}
\usepackage[margin=1in]{geometry}
\usepackage{graphicx}
\usepackage{amsmath}
\usepackage{afterpage}
% info for header block in upper right hand corner
\name{}
\class{Engineering 72}
\assignment{Homework 3}
\duedate{Due 10 February 2017}

\newcommand{\pn}[1]{\left( #1 \right)}
\newcommand{\abs}[1]{\left| #1 \right|}
\newcommand{\bk}[1]{\left[ #1 \right]}
\newcommand{\vc}[1]{\left\langle #1 \right\rangle}
\newcommand{\pder}[2]{\frac{\partial #1}{\partial #2}}

\newcommand{\norm}[1]{\lvert #1 \rvert}

% set numbering style for enumerated lists to be of form (a), (b), (c), etc.
\renewcommand{\labelenumi}{{(\alph{enumi})}}

% command to make 2in wide image centered on page
\newcommand{\diagram}[2]{\begin{center}\includegraphics[width=#2in, keepaspectratio]{#1}\end{center}}


\begin{document}
	
	\problemlist{1,2,3}
	
	\begin{problem}[1]
		Kirchhoff’s current law (KCL) can be used to write \textit{algebraic} equations for RLC circuits
		using the concept of impedance, which you learned in E59. If we think of impedance as a
		complex-valued resistance, then
		\begin{itemize}
			\item for a resistor, impedance $Z_R = R$
			
			\item for an inductor, impedance $Z_L = j\omega L$
			
			\item for a capacitor, impedance $Z_C = 1/(j\omega C)$.
		\end{itemize}
		Use KCL to write a set of equations in the form $A\textbf{x} = \textbf{b}$, where \textbf{x} is the vector of (complex)
		node voltages. Write a function to solve your system of equations using Matlab for an AC
		input with amplitude $V_{in} = 5$ V and frequency $f$ (note that in your system of equations this
		can be represented by the amplitude only). This circuit is a low-pass Butterworth filter with
		a cutoff frequency of 12 MHz. To see the effect of frequency on the output voltage, make
		a plot of log magnitude (dB) $= 20\log_{10}( \frac{V_{out}}{V_{in}} )$ (with $V_{out}$ as the magnitude of the complex
		voltage across the load resistor) versus frequency from 1 to 100 MHz.
		
		\diagram{1_Circuit}{5}
	\end{problem}
	
	\begin{solution}
		\vfill
	\end{solution}
	
	\begin{problem}[2]
		A heat sink for cooling computer chips is fabricated from copper with machined microchannels
		that carry a cooling fluid with temperature $T_f = 25^{\circ}$C. Assume that this is also the
		temperature of the walls of the channel (that is, convective heat transfer from the wall to the
		fluid is very effective). The lower edge of the heat sink is in contact with a thermal insulator
		that does not remove any heat. The figure shows a cross section of the heat sink, with the
		channels running into the page. Our goal is to compute the temperature distribution for the
		maximum allowable chip temperature of $T_c = 75^{\circ}$C and the flux of waste heat that can be
		removed.
		\diagram{Heat_sink}{4}
		\begin{enumerate}
			\item Instead of trying to compute the temperature distribution everywhere, we overlay a
			grid on the cross-section and determine the temperatures at a discrete set of points.
			Because of the symmetry in the problem, it is enough to compute the temperatures in the
			region covered by the points shown. The temperatures at the surface points are known
			(shown by open dots). The temperatures at the closed dots are unknown. Since this is a
			steady-state problem, the temperature distribution in any cross-section of the heat sink
			(into the page) is the same and the temperature at any grid point is approximately equal
			to the average of the temperatures at the four grid points connected to it. Use this
			approximation to set up a system of equations for the unknown temperatures. This is
			called a ``finite difference'' formulation of the problem. The grid points on the insulated
			surface require special treatment to enforce a no-flux condition. For example,
			$$ T_2 = \frac{T_1+T_4+T_4+T_f}{4} = \frac{T_1+2T_4+T_f}{4} $$
			(Effectively, we mirror $T_4$ on the bottom side of $T_2$. Come talk to us if you want to know
			why this is the correct method.) Make use of symmetry in a similar way when you
			think of the four grid points surrounding points on the left and right internal edges of
			the domain you are analyzing. Use Matlab to solve your system of equations for the
			unknown point temperatures.
			
			\item Use Matlab's \texttt{contourf} command to produce a graph of the steady-state temperature
			distribution. Include a printout of your graph with your homework. Use your graph to
			check if your temperature distribution seems reasonable.
		\end{enumerate}
	\end{problem}
	
	
	\begin{problem}[2 (cont.)]
		\begin{enumerate}
			\item[(c)] Fourier's law of heat conduction says that the flux of heat energy across a surface
			is proportional to the outward normal temperature gradient, and the constant of
			proportionality is the thermal conductivity, k, which for copper is 400 W/(m · K). In this
			situation, the heat transferred from the chip to the heat sink per unit length (in W per
			meter into the page) at the top surface (whose normal is aligned with the y direction) of
			the shaded domain we are analyzing is
			\begin{displaymath}
			\begin{split}
			Q_{\text{top}} &= k \int_{\text{left end of shaded domain}}^{\text{right end of shaded domain}} \frac{\partial T}{\partial y}dx \\
			&\approx k \cdot \frac{\Delta}{2} \left[ \frac{T_7-T_c}{\Delta} + 2\frac{T_8-T_c}{\Delta} + 2\frac{T_9-T_c}{\Delta} + \frac{T_{10}-T_c}{\Delta}\right] \\
			&= \frac{k}{2} \left[(T_7-T_c) + 2(T_8-T_c) + 2(T_9-T_c) + (T_{10}-T_c) \right]
			\end{split}
			\end{displaymath}
			In the equation above, $\Delta$ is the horizontal or vertical distance between any two adjacent
			grid points, the integral over the top edge of the domain is approximated using the
			trapezoidal rule, and the temperature gradient $\partial T/\partial y$ is approximated by the temperature
			difference between adjacent grid points, divided by $\Delta$. What is the rate of heat transfer
			(in W) from a 10 mm x 10 the chip to the heat sink? [Note that this calculation is an
			overestimate of the heat that can be removed for several reasons. The coarse grid limits
			the precision of the temperature distribution (so several steps of refining the grid until
			the further refinement does not significantly change the answer is warranted), we have idealized the heat transfer between the chip and the copper and between the copper and
			the fluid, and we assumed the fluid temperature stays constant as it flows into the page,
			but it would heat up. Extending the model to remove these approximations would be
			within your abilities if you needed a more reliable answer].
		\end{enumerate}
	\end{problem}
	
	\begin{solution}
		\vfill
	\end{solution}

	\newpage
	
	\begin{problem}[3]
		In this problem, we will reconsider the truss problem that we studied in class.
		\begin{enumerate}
			\item Recalculate the optimal location of joint $B$ that minimizes the sum of the bar lengths of
			the truss, this time using the \texttt{fminsearch} function in Matlab. What is that minimum
			total bar length? (By minimizing the total bar length, we are minimizing the weight and
			cost of the truss.) For this part of the problem, you will need to modify your \texttt{e72truss.m}
			code so that it outputs the total length of the truss, given a 2 $\times$ 1 vector representing the
			coordinates of the joint $B$ as input.
			
			\item Suppose there is an obstacle that requires $x \geq 3$, where $x$ is the $x$-coordinate of joint $B$.
			Calculate the optimal location of joint $B$ that minimizes the total bar lengths of the
			truss in this situation. What is that optimal length? You might be interested in Matlab’s
			\texttt{fmincon} function.
			
			\item Calculate the optimal location of joint $B$ that maximizes $\phi$, the max load per weight of
			the truss. What is the resulting optimal value of $\phi$ and the maximum load of the truss?
			
			\item Calculate the optimal location of joint $B$ that maximizes $\phi$, subject to the constraint
			$x \geq 3$. What is the resulting optimal value of $\phi$ and the maximum load of the truss?
		\end{enumerate}
		\textbf{Note}: If you want to know more about what’s happening with Matlab’s \texttt{fmincon}, you might
		look into using the \texttt{optimset} function to set values for the options \texttt{TolX} and \texttt{TolFun}. You’ll find
		more information in the Matlab documentation for \texttt{fmincon} function.

	\end{problem}
	
\end{document}