\documentclass[12pt,letterpaper]{hmcpset}
\usepackage[margin=1in]{geometry}
\usepackage{graphicx}
\usepackage{amsmath}
\usepackage{afterpage}
%\usepackage{indentfirst}

% info for header block in upper right hand corner
\name{}
\class{Engineering 72}
\assignment{Homework 4}
\duedate{Due 17 February 2017}

\newcommand{\pn}[1]{\left( #1 \right)}
\newcommand{\abs}[1]{\left| #1 \right|}
\newcommand{\bk}[1]{\left[ #1 \right]}
\newcommand{\vc}[1]{\left\langle #1 \right\rangle}
\newcommand{\pder}[2]{\frac{\partial #1}{\partial #2}}

\newcommand{\norm}[1]{\lvert #1 \rvert}

% set numbering style for enumerated lists to be of form (a), (b), (c), etc.
\renewcommand{\labelenumi}{{(\alph{enumi})}}

% command to make 2in wide image centered on page
\newcommand{\diagram}[2]{\begin{center}\includegraphics[width=#2in, keepaspectratio]{#1}\end{center}}


\begin{document}
	
	\problemlist{1,2,3}
	
	\begin{problem}[1]
		\setlength{\parskip}{3pt}
		A simple model of the vibration modes of a wing on a flying airplane is given below. It
		accounts for bending and twisting motion by modeling the wing as a mass attached to the
		aircraft body by a linear spring with constant $k_1$ and a torsional spring with constant $k_2$.
		Linear damping with constant $c_1$ and torsional damping with constant $c_2$ are not indicated in
		the figure but can also be considered.
		
		\diagram{Airplane}{5}
		
		The vertical motion $x(t)$ and rotational motion $\theta(t)$ of this system are governed by these
		linearized differential equations:
		\begin{displaymath}
		\begin{split}
		m\Ddot{x} - me\Ddot{\theta} + c_1 \Dot{x} + k_1 x &= 0\\
		(J + me^2)\Ddot{\theta} - me\Ddot{x} + c_2 \Dot{\theta} + k_2 \theta &= 0
		\end{split}
		\end{displaymath}
		Here, $m$ is the mass of the wing, $J$ is the moment of inertia of the wing about its center of
		mass, and $e$ is the horizontal distance between the center of mass and the effective connection
		point of the spring and damping elements.
		Use the following parameter values: $m =$ 10000 kg, $J =$ 5000 kg $\cdot$ m$^2$, $e =$ 1 m, $k_1 =$ 106 N/m
		and $k_2 =$ 105 Nm/rad.
		\begin{enumerate}
			\item Develop a set of four first-order differential equations for this system and write them in
			the matrix form $\Dot{\textbf{Y}} = A\textbf{Y} + \textbf{F}(t)$ where
			$$ \textbf{Y} = \begin{bmatrix}
				x\\
				\Dot{x}\\
				\theta\\
				\Dot{\theta}
				\end{bmatrix}
			$$
		\end{enumerate}
	\end{problem}
	
	\newpage
	
	\begin{problem}[1 (cont.)]
		\begin{enumerate}
		\item[(b)]  Suppose that there is no damping in the system ($c_1 = c_2 = 0$). What kind of motion
		characterizes the transient response of the system, and what are its vibrational frequencies?
		
		\item[(c)] Now suppose that there is no damping in the system ($c_1 = c_2 = 0$). What kind of motion
		characterizes the transient response of the system, and what are its vibrational frequencies?
		
		\item[(d)] Now suppose $c_1 = 10^5$ Ns/m and $c2 = 10^5$ Nm/(rad/s). What kind of motion characterizes
		the transient response of the system?
		\end{enumerate}
	\textbf{Hint:} Having a hard time computing eigenvalues of a 4 × 4 matrix? Use Matlab.

	\end{problem}
	
	%\begin{solution}
	%	\vfill
	%\end{solution}
	
	\afterpage{\null\newpage}
	
	
	\begin{problem}[2]
		\setlength{\parskip}{3pt}
		Producing chemicals in a batch process involves filling a vessel with liquid, sealing it, then
		heating it to a prescribed temperature. In the design of such a process it is important to
		calculate the time required for the liquid to reach the desired temperature. The vessel shown
		is a small model designed for quick processing. It has an electrical heating element contained
		within an inner metal jacket that has thermal resistance $R_{HL}$ (heater-to-liquid resistance).
		The thermal resistance of the vessel and its outer layer of insulation is $R_{La}$ (liquid-to-ambient
		resistance). The heater and liquid have thermal capacitances $C_H$ and $C_L$, respectively.
		
		All temperatures are to be measured relative to the ambient temperature. Initially both the
		heater and the liquid are at this ambient temperature with the heater turned off, so the heater
		(relative) temperature $\theta_H (t)$ and the liquid (relative) temperature $\theta_L (t)$ (which is assumed to
		be spatially uniform because there is a mixer in the vessel) have initial values of zero.
		
		The rate at which energy is supplied to the heating element is $q_i (t)$.
		
		\diagram{Heater}{2}
		
		At time $t = 0$ the heater is connected to an electrical source that supplies energy at a constant
		rate. The goal is to determine the response of the liquid temperature $\theta_L$ and to calculate the
		time required for the liquid to reach a desired temperature.
		
		The system of equations that governs this system is
		\begin{displaymath}
		\begin{split}
		\Dot{\theta}_H &= \frac{1}{C_H} \left[ q_i - q_{HL} \right]\\
		\Dot{\theta}_L &= \frac{1}{C_L} \left[ q_{HL} - q_{La} \right]
		\end{split} 
		\end{displaymath}
		where $q_{HL} = (\theta_H - \theta_L)/R_HL$ and $q_{La} = \theta_L / R_{La}$. Assuming the values 
		$C_H = 1 \times 10^3$ J/K,
		$C_L = (40/19) \times 10^4$ J/K,
		$R_{HL} = 1 \times 10^{−4}$	s $\cdot$ K/J, 
		$R_{La} = (19/21) × 10^{−4}$ s $\cdot$ K/J, and
		$q_i = 500$ kW, 
		find the liquid temperature as a function of time (suggestion: stick with the
		fractions, we're not being unnecessarily mean). What is the steady state temperature of the
		liquid (relative to ambient)? What is the time required for the liquid temperature to increase
		by 40 degrees?
	\end{problem}
	
	
	\begin{problem}[2 (cont.)]
		\setlength{\parskip}{3pt}
		\textbf{Note \#1:} If you would like to understand the nature of each of these quantities further, the
		analogous electrical circuit might be helpful. See \texttt{hw4-ElectricalThermalAnalogy.pdf} in
		the Homework folder on Sakai for an outline of the analogy, that circuit, and derivation of the
		equations.
		
		\textbf{Note \#2:} Most of the work on this problem should be done by hand. Be strategic about
		when you introduce numerical values into your work and when you use decimal numbers like 0.9047619 instead of exact expressions like 19/21. There are ways to avoid lots of fractions
		when diagonalizing your matrix.
	\end{problem}
	
	\begin{solution}
		\vfill
	\end{solution}
	
	\afterpage{\null\newpage}
	\newpage
	
	
	
	\begin{problem}[3]
		To reduce the possibility of damage caused by impulsive voltage spikes (due to surges from
		equipment or lightning), the following electrical filter is installed in the power line for some
		sensitive equipment.
		
		\diagram{Circuit}{3}
		
		The values of the filter components are $L_1 = L_2 = 10$ mH, $C = 1$ $\mu$F. The load resistance is 50 $\Omega$.
		\begin{enumerate}
			\item Derive a set of three first-order differential equations for this system in terms of the
			variables $v_R$, $v_C$ and $i_C$ and their derivatives, where $v_C$ is the voltage across the capacitor
			and $i_C$ is the current through the capacitor. The voltage spike is described by the
			function $v_s (t)$. Explain the governing principle(s) behind each differential equation.
			Write the equations in the matrix form $\Dot{\textbf{Y}} = A\textbf{Y} + \textbf{F}(t)$ where
			$$ \textbf{Y} = \begin{bmatrix}
			v_R\\
			v_C\\
			i_C
			\end{bmatrix} $$
			
			\item Compute $v_R (t)$, the output system voltage, in the case where $v_s (t) = 10^{−3} \delta(t)$. This
			function models a voltage spike with amplitude 100 V and duration 10 $\mu$s. Plot $v_R (t)$ and determine the maximum system voltage that is reached.
		\end{enumerate}
		
		\textbf{Hint:} You do not need to compute the entire matrix exponential. Think about what
		parts of the matrix exponential you really need to get the information you want.
	\end{problem}

	
	\begin{solution}
		\vfill
	\end{solution}


\end{document}