\documentclass[12pt,letterpaper]{hmcpset}
\usepackage[margin=1in]{geometry}
\usepackage{graphicx}
\usepackage{amsmath}
\usepackage{afterpage}
%\usepackage{indentfirst}

% info for header block in upper right hand corner
\name{}
\class{Engineering 72}
\assignment{Homework 5}
\duedate{Due 24 February 2017}

\newcommand{\pn}[1]{\left( #1 \right)}
\newcommand{\abs}[1]{\left| #1 \right|}
\newcommand{\bk}[1]{\left[ #1 \right]}
\newcommand{\vc}[1]{\left\langle #1 \right\rangle}
\newcommand{\pder}[2]{\frac{\partial #1}{\partial #2}}

\newcommand{\norm}[1]{\lvert #1 \rvert}

% set numbering style for enumerated lists to be of form (a), (b), (c), etc.
\renewcommand{\labelenumi}{{(\alph{enumi})}}

% command to make 2in wide image centered on page
\newcommand{\diagram}[2]{\begin{center}\includegraphics[width=#2in, keepaspectratio]{#1}\end{center}}


\begin{document}
	
	\problemlist{1,2,3}
	
	\begin{problem}[1]
		In some periodic vibratory systems, external energy is supplied to the system over part of a
		period and dissipated within the system in another part of the period. This involves nonlinear
		damping and is known as a relaxation oscillation or a van der Pol oscillation. Van der Pol
		was a telecommunications engineer who discovered these stable oscillations while building
		electronic circuit models of the human heart. The model has since been used to describe
		many other physical phenomena, which you can search on the internet to see.
		
		The van der Pol equation is
		$$\ddot{x} + \mu (x^2 - 1)\dot{x} + x = 0$$
		where $\mu$ is a parameter to be varied and $x$ is the pacemaker signal in the cardiac model case.
		\begin{enumerate}
			\item Write the equation as a system of first order equations, determine the equilibrium point(s)
			of the system, linearize about each equilibrium point and characterize the stability of the
			system for $-3 < \mu < 3$.
			
			\item The interesting van der Pol results are for $\mu > 0$. Use Matlab to create separate phase
			portraits for $\mu = 0, 1$ and 3. Comment on the relation between these plots and the
			expectations from part (a).
		\end{enumerate}
		\textbf{Some hints} if you are using Matlab:
		\begin{itemize}
			\item The \texttt{quiver} command shows both the magnitude and the direction at each point
			in the vector field. Sometimes the scaling is set by large magnitude vectors in
			a way that makes small ones impossible to see. To have it show direction only,
			normalize all of the vectors. Instead of using a command like \texttt{quiver(x,y,u,v)},
			use \texttt{quiver(x,y,u./L,v./L)} where \texttt{L=sqrt(u.\^{}2+v.2)}
			
			\item You can add streamlines to the quiver plots to help you visualize the trajectories. For
			example try \texttt{startx = 0.5}; \texttt{starty = 0}; \texttt{streamline(x,y,u,v,startx,starty)}
		\end{itemize}
	\end{problem}
	
	
	\begin{solution}
		\vfill
	\end{solution}
	
	\newpage
	
	\begin{problem}[2]
		A first step in evaluating the dynamics of wheeled vehicles such as bicycles and motorcycles is
		understanding the stability of a rolling disk. This disk rolls without slipping on a horizontal
		plane. Its angular velocity is characterized by $\dot{\alpha}$, the rate of change of the lean angle $\alpha$, $\Omega$ the
		rate of spin about the vertical \textbf{z} direction, and $\omega$ the rate of spin about its axial direction.
		The equations governing the orientation are
		\begin{displaymath}
			\begin{split}
				(2k+1)\dot{\omega} + \dot{\alpha}\Omega \sin\alpha &= 0\\
				k\Omega^2 \sin\alpha \cos\alpha + (2k+1)\omega\Omega \sin\alpha - (k+1)\ddot{\alpha} &= \frac{g}{r} \cos\alpha\\
				\dot{\Omega} \sin\alpha + 2\dot{\alpha} \Omega\cos\alpha + 2\omega\dot{\alpha} &= 0
			\end{split}
		\end{displaymath}
		
		where $r$ is the radius of the disk and $g$ is the gravitational constant. For a solid disk as we are
		considering $k = \frac{1}{4}$.
		
		Arbitrary motion may be simulated with these equations, but in this problem we are interested
		in equilibrium states of the system and their stability. Equilibrium states of this system
		correspond to steady motion of the disk. You may want to check for yourself that all
		equilibrium states of the system must have $\alpha = \alpha_0$, $\dot\alpha = 0$, $\omega = \omega_0$, and $\Omega = \Omega_0$, where $\alpha_0$, $\omega_0$, and $\Omega_0 $ are constants that satisfy a nonlinear equation. In particular, we're interested in two
		particular equilibrium states with $\alpha_0 = \pi/2$, which corresponds to the disc being vertical.
		
		\diagram{Disk}{3}
		
		\begin{enumerate}
			\item If $\Omega_0 = 0$ the disk rolls along a straight line and $\omega_0$ is the rolling angular velocity. For
			what angular velocities $\omega_0$ does local stability analysis predict unstable motion? (You
			answer will be in terms of $r$ and $g$.)
			
			\item If $ \omega_0 = 0 $ the disk spins in place about the vertical axis with angular velocity $\Omega_0$. For
			what angular velocities $\Omega_0$ does local stability analysis predict unstable motion?
		\end{enumerate}
		There are plenty of other interesting cases (rolling in a circle, various precessions). If you
		are interested come discuss some time, or take E175 (Dynamics of Rigid Bodies).
	\end{problem}
	
	\begin{solution}
		\vfill
	\end{solution}
	
	\newpage
	
	\begin{problem}[3]
		Three weeks after teaching at Harvey Mudd College
		began in the fall of 1957, the Soviet Union launched
		the first-ever artificial Earth satellite: the spherically shaped
		Sputnik I. They launched Sputnik II the
		following month. There was a lot of pressure for
		the US to keep up, and in January 1958 the US
		launched Explorer I. Explorer I was very different
		looking---long and narrow---and it was supposed to
		rotate about its own long axis centerline.
		A radio astronomer named Ronald Bracewell had tracked and analyzed the flight of Sputnik I
		and understood its stable behavior. He also knew that Explorer I would not be stable and
		would tumble end over end. He tried to reach engineers at JPL to warn them but security
		concerns prevented him from getting his information to the right people and he was only
		able to share it by publishing it in Nature later in 1958. Too late--Explorer I made just one
		stably-spinning orbit of the Earth then began to tumble.
		It was not unreasonable for the JPL engineers to plan for spin about the long axis. This would
		be spinning about the axis with the minimum moment of inertia, which should be stable.
		
		However once in flight, its flexible antennas were deployed, and their vibration dissipated
		energy and the axis changed. We're not going to analyze the transition, but we can study the
		effect of rotation about different axes.
		
		The equations of motion governing the orientation for a spinning rigid body with at least
		two axes of symmetry are
		\begin{displaymath}
			\begin{split}
				0 &= I_x \dot{\omega}_x + (I_z - I_y)\omega_z\omega_y\\
				0 &= I_y \dot{\omega}_y + (I_x - I_z)\omega_x\omega_z\\
				0 &= I_z \dot{\omega}_z + (I_y - I_x)\omega_y\omega_x\\
			\end{split}
		\end{displaymath}
		where $\omega_x$, $\omega_y$ and $\omega_z$ are the angular velocities of the body about the $x$-, $y$- and $z$-axes,
		respectively, and $I_x$, $I_y$ and $I_z$ are the corresponding central moments of inertia of the body
		about each axis.
		\begin{enumerate}
			\item Show that constant rotation of the body along each of the axes is an equilibrium state of
			the differential equations above. In other words, show that $(\omega_x$, $\omega_y$, $\omega_z)=(c$, 0, 0) or (0, $c$, 0) or (0, 0, $c$), where $c$ is a constant, is an equilibrium state. What information does
			the local stability analysis provide about these equilibrium states? Explain why local
			stability analysis does not fully explain the stability of the three equilibrium states.
			
			\item Use Matlab to investigate the stability of each of three types of equilibrium states.
			The moment of inertia about the long axis is $I_x = 0.06$ kg $\cdot$ m$^2$, and the other two
			moments of inertia, while nominally the same, will be shifted by just 1\% to model a
			redistribution due to antenna flexibility: $I_y = 0.099$ kg $\cdot$ m$^2$ and $I_z = 0.101$ kg $\cdot$ m$^2$. Write
			a few sentences about how you carried out your investigation and whether your work
			demonstrates the expected instability. Include figures that best explain your claims.
		\end{enumerate}
	\end{problem}
	
	\newpage
	
	\begin{problem}[3 (cont.)]
		\textbf{More history}: In addition to not having Matlab or OpenRocket in 1958, here is another
		detail to give context to the work being done at that time. ``During the 1940s and 1950s,
		JPL used the word `computer' to refer to a person rather than a machine. The all-female
		computer team, many of the members recruited right out of high school, were responsible
		for doing all the math by hand required to plot satellite trajectories and more.'' (from
		\texttt{http://www.nasa.gov/mission\_pages/explorer/computers.html})
	\end{problem}
	
	
\end{document}
